%%%%%%%%%%%%%%%%%%%%%%%%%%%%%%%%%%%%%%%%%%%%%%%%%%%%%%%%%%%%%%%%%%%%%%
%
%.IDENTIFICATION $Id: templatelarge.tex.src,v 1.29 2008/01/25 10:47:12 fsogni Exp $
%.LANGUAGE       TeX, LaTeX
%.ENVIRONMENT    ESOFORM
%.PURPOSE        Template application form for ESO Observing time.
%.AUTHOR         The Esoform Package is maintained by the Observing
%                Programmes Office (OPO) while the background software
%                is provided by the User Support System (USS) Department.
%
%-----------------------------------------------------------------------
%
%
%                   ESO LA SILLA PARANAL OBSERVATORY
%                   --------------------------------
%                   LARGE PROGRAMME PHASE 1 TEMPLATE
%                   --------------------------------
%
%
%
%          PLEASE CHECK THE ESOFORM USERS' MANUAL FOR DETAILED 
%              INFORMATION AND DESCRIPTIONS OF THE MACROS. 
%     (see the file usersmanual.tex provided in the ESOFORM package) 
%
%
%        ====>>>> TO BE SUBMITTED THROUGH WEB UPLOAD  <<<<====
%               (see the Call for Proposals for details)
%
%%%%%%%%%%%%%%%%%%%%%%%%%%%%%%%%%%%%%%%%%%%%%%%%%%%%%%%%%%%%%%%%%%%%%%

%%%%%%%%%%%%%%%%%%%%%%%%%%%%%%%%%%%%%%%%%%%%%%%%%%%%%%%%%%%%%%%%%%%%%%
%
%                      I M P O R T A N T    N O T E
%                      ----------------------------
%
% By submitting this proposal, the Principal Investigator takes full
% responsibility for the content of the proposal, in particular with
% regard to the names of CoI's and the agreement to act in accordance
% with the ESO policy and regulations, should observing time be
% granted.
%
%%%%%%%%%%%%%%%%%%%%%%%%%%%%%%%%%%%%%%%%%%%%%%%%%%%%%%%%%%%%%%%%%%%%%% 

%
%    - LaTeX *is* sensitive towards upper and lower case letters.
%    - Everything after a `%' character is taken as comments.
%    - DO NOT CHANGE ANY OF THE MACRO NAMES (words beginning with `\’).
%    - DO NOT INSERT ANY TEXT OUTSIDE THE PROVIDED MACROS.
%

%
%    - All parameters are checked at the verification or submission.
%    - Some parameters are also checked during the pdfLaTeX
%      compilation.  If this is not the case, this is indicated by the
%      phrase
%      "This parameter is NOT checked at the pdfLaTeX compilation."
%

\documentclass{esoformlarge}

% The list of LaTeX definitions of commonly used astronomical symbols
% is already included in the style file common2e.sty (see Table 1 in
% the Users' Manual).  If you have your own macros or definitions,
% please insert them here, between the \documentclass{esoformlarge}
% and the \begin{document} commands.
%
%     PLEASE USE NEITHER YOUR OWN MACROS NOR ANY TEX/LATEX MACROS  
%       IN THE \Title, \Abstract, \PI, \CoI, and \Target MACROS.
%
% WARNING: IT IS THE RESPONSIBILITY OF THE APPLICANTS TO STAY WITHIN THE
% CURRENT BOX LIMITS AND ELIMINATE POTENTIAL OVERFILL/OVERWRITE PROBLEMS. 

\begin{document}

%%%%%%%%%%%%%%%%%%%%%%%%%%%%%%%%%%%%%%%%%%%%%%%%%%%%%%%%%%%%%%%%%%%%%%%%
%%%%% CONTENTS OF THE FIRST PAGE %%%%%%%%%%%%%%%%%%%%%%%%%%%%%%%%%%%%%%%
%%%%%%%%%%%%%%%%%%%%%%%%%%%%%%%%%%%%%%%%%%%%%%%%%%%%%%%%%%%%%%%%%%%%%%%%
%
%---- BOX 1 ------------------------------------------------------------
%
% You should use this template for period 104A applications ONLY.
%
% DO NOT EDIT THE MACRO BELOW. 

\Cycle{104A}

% Type below, within the curly braces {}, the title of your observing
% programme (up to two lines).
% This parameter is NOT checked at the pdfLaTeX compilation.
%
% DO NOT USE ANY TEX/LATEX MACROS IN THE TITLE.

\Title{This Is The Proposal Title This Is The Proposal Title}  

% Type below the numeric code corresponding to the subcategory of your
% programme. Please note that starting in P101 there are major changes
% to the subcategories in categories A and B. Please check the updated
% categories at
% https://www.eso.org/sci/observing/phase1/opc-categories.html
% and the ESOFORM User Manual.

\SubCategoryCode{X0}   

%This is the only type valid with this template. Do NOT change it.
\ProgrammeType{LARGE}

% For GTO proposals only: uncomment the following and fill out the GTO
% programme code (as communicated to the respective GTO coordinator).

%\GTOcontract{INS-consortium}

%---- BOX 2 ------------------------------------------------------------
%
% Type below a concise abstract of your proposal (up to 13 lines).
% This parameter is NOT checked at the pdfLaTeX compilation.
%
% DO NOT USE ANY TEX/LATEX MACROS IN THE ABSTRACT.

\Abstract{This is a concise abstract of the proposal which may have up
  to 13 lines.}

%---- BOX 3 ------------------------------------------------------------
%
% Description of the observing run(s) necessary for the completion of
% your programme.  The macro takes ten parameters: run ID, period,
% instrument, time requested, month preference, moon requirement,
% seeing requirement, transparency requirement, observing mode and 
% run type.
%
% 1. RUN ID
% Valid values: A, B, ..., Z
%
% 2. PERIOD
% For Paranal and APEX instruments four periods are allowed,
% hence the valid range is from 104 to 107.
% La Silla large proposals can request up to Period 97.
% This parameter is NOT checked at the pdfLaTeX compilation.
%
% 3. INSTRUMENT
% Valid values: ARTEMIS ESPRESSO-1UT FLAMES FORS2 GRAVITY HARPS HAWKI KMOS MATISSE MUSE OMEGACAM PIONIER SEPIA SPHERE SpecialNTT UVES VIRCAM VISIR XSHOOTER
%
% 4. TIME REQUESTED
% In hours for Service Mode, in nights for Visitor Mode.
% In either case the time can be rounded up to  1 decimal place.    
% This parameter is NOT checked at the pdfLaTeX compilation.
% 
% 5. MONTH PREFERENCE
% Valid values: oct, nov, dec,
% jan, feb, mar, apr,
% may, jun, jul, aug,
% sep, any
%
% 6. MOON REQUIREMENT
% Valid values: d, g, n
%
% 7. SEEING REQUIREMENT
% Valid values: 0.4, 0.6, 0.8, 1.0, 1.2, 1.4, n
%
% 8. TRANSPARENCY REQUIREMENT
% Valid values: CLR, PHO, THN
%
% 9. OBSERVING MODE
% Valid values: v, s
% 
% 10. RUN TYPE
% Valid values: TOO 
% Please leave blank for all Large Programme proposals, except if the run is a TOO run.
% In this case, please specify.
% If the field is left blank, a default, normal, non-TOO run is assumed.
% If a TOO run is specified please make sure that you fill in the TOO page.

\ObservingRun{A}{104}{FORS2}{40h}{nov}{n}{0.8}{PHO}{s}{}
\ObservingRun{A/alt}{104}{FORS2}{8n=3x2+4H2}{nov}{n}{0.8}{PHO}{v}{}
\ObservingRun{B}{104}{FORS2}{6n=6x1}{dec}{n}{0.6}{CLR}{v}{}
\ObservingRun{C}{104}{HARPS}{8n}{feb}{n}{0.8}{THN}{v}{}
\ObservingRun{D}{104}{FORS2}{1.5n}{mar}{n}{0.8}{THN}{v}{}

% Proprietary time requested.
% Valid values: % 0, 1, 2, 6, 12

\ProprietaryTime{12}

%---- BOX 4 ------------------------------------------------------------
% Please provide the ESO User Portal username for the Principal 
% Investigator (PI) in the \PI field.
%
% For the Co-I's (CoI) please fill in the following details:
% First and middle initials, family name, the institute code
% corresponding to their affiliation.
% The corresponding affiliation should be entered for EACH
% Co-I separately in order to ensure the correct details of 
% all Co-I's are stored in the ESO database.
% You can find all institute codes listed according to country
% on the following webpage: 
% http://www.eso.org/sci/observing/phase1/countryselect.html
%
% DO NOT USE ANY TEX/LATEX MACROS HERE.
%

\PI{JSMITH999} 
% Replace with PI's ESO User Portal username.

\CoI{L.}{Ma\c con}{1098}
\CoI{R.}{Men\'endez}{1098}
\CoI{S.}{Bailer-Brown}{1098}
\CoI{K.L.}{Giorgi}{1098}
\CoI{S.}{Lichtman}{1258}

% Please note:
% Due to the way in which the proposal receiver system parses
% the CoI macro, the number of pairs of curly brackets '{}'
% in this macro MUST be strictly equal to 3, i.e., the
% number of parameters of the macro. Accordingly, curly
% brackets should not be used within the parameters (e.g.,
% to protect LaTeX signs).
%
% For instance:
% \CoI{L.}{Ma\c con}{1098}
% \CoI{R.}{Men\'endez}{1098}
%
% are valid, while
%
% \CoI{L.}{Ma{\c}con}{1098}
% \CoI{R.}{Men{\'}endez}{1098}
%
% are not. Unfortunately the receiver does not give an
% explicit error message when such invalid forms are
% used in the CoI macro, but the processing of the proposal
% keeps hanging indefinitely.

%%%%%%%%%%%%%%%%%%%%%%%%%%%%%%%%%%%%%%%%%%%%%%%%%%%%%%%%%%%%%%%%%%%%%%%%
%%%%% THE THREE PAGES OF THE SCIENTIFIC DESCRIPTION %%%%%%%%%%%%%%%%%%%%
%%%%%%%%%%%%%%%%%%%%%%%%%%%%%%%%%%%%%%%%%%%%%%%%%%%%%%%%%%%%%%%%%%%%%%%%
%
%---- BOX 5 ------------------------------------------------------------
%
%               THIS DESCRIPTION IS RESTRICTED TO THREE PAGES 
%
%   THE RELATIVE LENGTHS OF EACH OF THESE FIVE SECTIONS ARE VARIABLE,
%                BUT THEIR SUM IS RESTRICTED TO THREE PAGES
%
% All macros in this box are NOT checked at the pdfLaTeX compilation.

\ScientificRationale{Scientific rationale: scientific background of
  the project, pertinent references; previous work plus justification
  for present proposal.}

\ImmediateObjective{Immediate objective of the proposal: state what is
  actually going to be observed and what shall be extracted from the
  observations, so that the feasibility becomes clear.}

\TelescopeJustification{Justification for the use of the selected
  telescope (e.g., VLT, NTT, etc...)  with respect to other available
  alternatives.}

\ModeJustification{Explain if a particular observing mode is specifically needed for this programme. If either can, in principle, be used then please enter N/A.}

%%%%%%%%%%%%%%%%%%%%%%%%%%%%%%%%%%%%%%%%%%%%%%%%%%%%%%%%%%%%%%%%%%%%%%%%
%%%%% THE TWO PAGES OF THE FIGURES %%%%%%%%%%%%%%%%%%%%%%%%%%%%%%%%%%%%%
%%%%%%%%%%%%%%%%%%%%%%%%%%%%%%%%%%%%%%%%%%%%%%%%%%%%%%%%%%%%%%%%%%%%%%%%
%
% Up to TWO pages of figures can be added to your proposal.  If you
% use color figures, make sure that they are still readable if printed
% in black and white.  Figures must be in PDF or JPEG format.
% Each figure has a size limit of 1MB.

\MakePicture{galaxy.pdf}{angle=90,width=10cm}
\MakeCaption{Fig.~1: A caption for your figure can be inserted here.}

%%%%%%%%%%%%%%%%%%%%%%%%%%%%%%%%%%%%%%%%%%%%%%%%%%%%%%%%%%%%%%%%%%%%%%%%
%%%%% THE PAGE OF BOXES 6, 7, AND 8 %%%%%%%%%%%%%%%%%%%%%%%%%%%%%%%%%%%%
%%%%%%%%%%%%%%%%%%%%%%%%%%%%%%%%%%%%%%%%%%%%%%%%%%%%%%%%%%%%%%%%%%%%%%%%
%
%---- BOX 6 ------------------------------------------------------------
%
% Indicate below the experience of the applicants with telescopes,
% instrumentation, data reduction and delivery of data products to the 
% ESO Archive.
% This macro is NOT checked at the pdfLaTeX compilation.

\Experience{Indicate here the experience of the applicants with telescopes, instrumentation, data reduction and their track record for submission of data products to the ESO Archive from previous large programmes. The PI should describe the data quality assessment process and the data reduction for the production of science data products.}

%---- BOX 7 ------------------------------------------------------------
%
% Indicate below the strategy for data reduction and analysis with
% description of the available resources to the observing team, such
% as: computing capabilities, research assistants, etc...
% In addition a delivery plan for data products should be described here.
% This macro is NOT checked at the pdfLaTeX compilation.

\Resources{Indicate here the strategy for data reduction and analysis including a description of the available resources to the team. PIs should provide the list of science data products planned to be published, through Phase 3, in the ESO archive and their legacy value. PIs must provide a timeline for the delivery plan, which should include at least one data release within the duration of the LP and must be finalised within two years of the completion of data acquisition. }

%---- BOX 8 ------------------------------------------------------------
%
% Take advantage of this box to provide any special remark (up to ten
% lines).  
% This macro is NOT checked at the pdfLaTeX compilation.

\SpecialRemarks{Take advantage of this box to provide any special
  remark (up to ten lines).}

%%%%%%%%%%%%%%%%%%%%%%%%%%%%%%%%%%%%%%%%%%%%%%%%%%%%%%%%%%%%%%%%%%%%%%%%
%%%%% THE PAGE OF BOXES 9 AND 10 %%%%%%%%%%%%%%%%%%%%%%%%%%%%%%%%%%%%%%%
%%%%%%%%%%%%%%%%%%%%%%%%%%%%%%%%%%%%%%%%%%%%%%%%%%%%%%%%%%%%%%%%%%%%%%%%
%
%---- BOX 9 ------------------------------------------------------------
%
% Provide below a careful justification of the requested lunar phase
% and of the requested number of nights or hours. Provide all
% information necessary to reproduce your ETC calculations. When
% applicable, specify the magnitude system employed (e.g., Vega, AB). 
% All macros in this box are NOT checked at the pdfLaTeX compilation.

\WhyLunarPhase{Provide here a careful justification of the requested
  lunar phase.}  

\WhyNights{Provide here a careful justification of the requested
  number of nights or hours.  ESO Exposure Time Calculators exist for
  all Paranal and La Silla instruments and are available at
  the following web address: http://www.eso.org/observing/etc .
  Links to exposure time calculators for APEX instrumentation
  can be found in Section 7 of the Call for Proposals.}

% Please specify the type of calibrations needed.
% Valid values: standard, special
% In case of special calibration the second parameter specifies them.

\Calibrations{special}{Adopt a special calibration}
%\Calibrations{standard}{}

%---- BOX 10 -----------------------------------------------------------
% 
% Use of the ESO facilities during the last 2 years (4 observing
% periods) and description of the status of the obtained data.
% This macro is NOT checked at the pdfLaTeX compilation.

\LastObservationRemark{Report on the use of the ESO facilities during
  the last 2 years (4 observing periods). Describe the status of the
  data obtained and the scientific output generated.}

%---- BOX 11 -----------------------------------------------------------
%
% Applicant's publications related to the subject of this proposal
% during the past two years.  Use the simplified abbreviations for
% references as in A&A.  Separate each reference with the following
% usual LaTex command: \smallskip\\
%   
%   Name1 A., Name2 B., 2001, ApJ, 518, 567: Title of article1
%   \smallskip\\
%   Name3 A., Name4 B., 2002, A\&A, 388, 17: Title of article2
%   \smallskip\\
%   Name5 A. et al., 2002, AJ, 118, 1567: Title of article3
%
% This macro is NOT checked at the pdfLaTeX compilation.

\Publications{
  Name1 A., Name2 B., 2001, ApJ, 518, 567: Title of article1
  \smallskip\\
  Name3 A., Name4 B., 2002, A\&A, 388, 17: Title of article2
  \smallskip\\
  Name5 A. et al., 2002, AJ, 118, 1567: Title of article3
}

%%%%%%%%%%%%%%%%%%%%%%%%%%%%%%%%%%%%%%%%%%%%%%%%%%%%%%%%%%%%%%%%%%%%%%%%
%%%%% THE PAGE OF THE TARGET/FIELD LIST %%%%%%%%%%%%%%%%%%%%%%%%%%%%%%%%
%%%%%%%%%%%%%%%%%%%%%%%%%%%%%%%%%%%%%%%%%%%%%%%%%%%%%%%%%%%%%%%%%%%%%%%%
%
%---- BOX 12 -----------------------------------------------------------
%
% Complete list of targets/fields requested.  The macro takes nine 
% parameters: run ID, target field/name, RA, Dec, time on target, magnitude, diameter,
% additional information, reference star.
%
% 1. RUN ID
% Valid values: run IDs specified in BOX 3
% 
% 2. Target field/Name
%
% 3. RA (J2000)
% Format: hh mm ss.f
% Use 00 00 00 for unknown coordinates
% This parameter is NOT checked at the pdfLaTeX compilation.
% 
% 4. Dec (J2000)
% Format: dd mm ss
% Use 00 00 00 for unknown coordinates
% This parameter is NOT checked at the pdfLaTeX compilation.
%
% 5. TIME ON TARGET
% Format: hours (overheads and calibration included)
% This parameter is NOT checked at the pdfLaTeX compilation.
%
% 6. MAGNITUDE
% This parameter is NOT checked at the pdfLaTeX compilation.
%
% 7. ANGULAR DIAMETER
% This parameter is NOT checked at the pdfLaTeX compilation.
%
% 8. ADDITIONAL INFORMATION
% Any relevant additional information may be inserted here.
% For APEX runs, the requested PWV and the acceptable LST range
%     MUST be specified here for each target. 
% This parameter is NOT checked at the pdfLaTeX compilation.
%
% 9. REFERENCE STAR ID
% See Users' Manual.
% This parameter is NOT checked at the pdfLaTeX compilation.
%
% Long lists of targets will continue on the last page of the
% proposal.
%
%
%                       ** VERY IMPORTANT ** 
% The scheduling of your programme will take into account ALL targets.
%
% DO NOT USE ANY TEX/LATEX MACROS FOR THE TARGETS.

\Target{ABD}{NGC 104}{00 24 06}{-72 04 58}{3.0}{5}{30 min}{47 Tuc}{}
\Target{A}{NGC 253}{00 47 33.1}{-25 17 17.8}{10.0}{8}{}{Seyfert gal.}{}
\Target{BC}{NGC 1851}{05 14 06.3}{-40 02 50}{8.0}{8.8}{}{glob. cluster}{}
\Target{B}{NGC 1316}{03 22 41.5}{-37 12 33}{15.0}{9.7}{10 min}{Fornax  A}{}
\Target{B}{NGC 1365}{03 33 36}{-36 08 27}{15.0}{10}{}{Seyfert gal.}{}
\Target{C}{M 42}{05 35.3}{-05 23.5}{2.0}{4}{1 deg}{}{}
\Target{C}{Rosette}{06 33.7}{+04 59.9}{3.0}{}{1 deg}{NGC 2237}{}
\Target{D}{NGC 2997}{09 45 38}{-31 11 25}{10.0}{}{}{Sc galaxy}{S133231219553}

%                      *****************
%                      ** PWV limits **
% For all APEX instruments users must specify the PWV upper
% limits for each target. For example:
%\Target{}{Alpha CMa}{06 45 08.9}{-16 42 58}{1}{-1.4}{6 mas}{PWV=1.0mm, Sirius}{}
%\Target{}{HD 104237}{12 00 05.6}{-78 11 33}{1}{}{}{PWV<0.7mm;LST=9h00-15h00}{}
%                      *****************

% Use TargetNotes to include any comments that apply to several or all
% of your targets.
% This macro is NOT checked at the pdfLaTeX compilation.

\TargetNotes{A note about the targets and/or strategy of selecting the targets during the run. For APEX runs please remember to specify the PWV limits for each target under Additional info in the table above.}

%%%%%%%%%%%%%%%%%%%%%%%%%%%%%%%%%%%%%%%%%%%%%%%%%%%%%%%%%%%%%%%%%%%%%%%
%---- BOX 12a -- ESO Archive ------------------------------------------
%%%%%%%%%%%%%%%%%%%%%%%%%%%%%%%%%%%%%%%%%%%%%%%%%%%%%%%%%%%%%%%%%%%%%%%
% Are the data requested in this proposal on the ESO Archive
% (http://archive.eso.org)? If yes, explain the need for new data.
% This macro is NOT checked at the pdfLaTeX compilation.

\RequestedDataRemark{Are the data requested in this proposal in the
  ESO Archive (http://archive.eso.org)? If yes, explain the need for
  new data.}

%%%%%%%%%%%%%%%%%%%%%%%%%%%%%%%%%%%%%%%%%%%%%%%%%%%%%%%%%%%%%%%%%%%%%
%---- BOX 12b -- ESO GTO/Public Survey Programme Duplications--------
%%%%%%%%%%%%%%%%%%%%%%%%%%%%%%%%%%%%%%%%%%%%%%%%%%%%%%%%%%%%%%%%%%%%%
% If any of the targets/regions in ongoing GTO Programmes and/or 
% Public Surveys are being duplicated here, please explain why.

\RequestedDuplicateRemark{
  Specify whether there is any duplication of targets/regions covered
  by ongoing GTO and/or Public Survey programmes. If so, please
  explain the need for the new data here. Details on the protected
  target/fields in these ongoing programmes can be found at:

  GTO programmes: http://www.eso.org/sci/observing/teles-alloc/gto.html

  Public surveys:
  http://www.eso.org/sci/observing/PublicSurveys/sciencePublicSurveys.html
}

%%%%%%%%%%%%%%%%%%%%%%%%%%%%%%%%%%%%%%%%%%%%%%%%%%%%%%%%%%%%%%%%%%%%%%%%
%%%%% THE PAGE OF SCHEDULING REQUIR. AND INSTRUMENT CONFIGURATIONS %%%%%
%%%%%%%%%%%%%%%%%%%%%%%%%%%%%%%%%%%%%%%%%%%%%%%%%%%%%%%%%%%%%%%%%%%%%%%%
%
%---- BOX 13 -----------------------------------------------------------
%
% 1. RUN SPLITTING, FOR A GIVEN ESO TELESCOPE (Visitor Mode only)
%

% This line should remain uncommented if the proposal involves 
% time-critical observations, or observations to be performed at specific 
% time intervals. Please leave these brackets blank. Details of time 
% constraints can be entered in Special Remarks and using the 
% other flags in Box 13.
%
\HasTimingConstraints{}

% 1st argument: run ID
% Valid values: run IDs specified in BOX 3
%
% 2nd argument: run splitting requested for sub-runs
% This parameter is NOT checked at the pdfLaTeX compilation.

\RunSplitting{B}{2,10s,2,20w,2}
\RunSplitting{C}{2,10s,2,20w,2,15s,4H2}

% 2. SPECIFIC DATE(S) FOR TIME-CRITICAL OBSERVATIONS
% Please note: The dates must correspond to the LOCAL CHILEAN observing dates.
%
% 1st argument: run ID
% Valid values: run IDs specified in BOX 3
%
% 2nd argument: Chilean start date for the critical period
% Format: dd-mmm-yyyy
% This parameter is NOT checked at the pdfLaTeX compilation.
%
% 3rd argument: Chilean end date for the critical period
% Format: dd-mmm-yyyy
% This parameter is NOT checked at the pdfLaTeX compilation.
%
% 4th argument: Reason for the time-critical dates specified.
% 

\TimeCritical{A}{12-nov-19}{14-nov-19}{Insert reason for time-critical observations.}
\TimeCritical{D}{12-nov-19}{14-nov-19}{Insert reason for time-critical observations.}

% 3. UNSUITABLE PERIOD(S) OF TIME
%
% 1st argument: run ID
% Valid values: run IDs specified in BOX 3
%
% 2nd argument: Chilean start date for the unsuitable time
% Format: dd-mmm-yyyy
% This parameter is NOT checked at the pdfLaTeX compilation.
%
% 3rd argument: Chilean end date for the unsuitable time
% Format: dd-mmm-yyyy
% This parameter is NOT checked at the pdfLaTeX compilation.

\UnsuitableTimes{A}{15-jan-20}{18-jan-20}{Insert explanation of unsuitable time here.}
\UnsuitableTimes{B}{15-jan-20}{18-jan-20}{Insert explanation of unsuitable time here.}
\UnsuitableTimes{C}{20-jan-20}{23-jan-20}{Insert explanation of unsuitable time here.}

% 4. LINK FOR COORDINATED OBSERVATIONS BETWEEN DIFFERENT RUNS
%
% 1st argument: run ID
% Valid values: run IDs specified in BOX 3
%
% 2nd argument: relationship
% Valid value: after, simultaneous
%
% 3rd argument: run ID
% Valid values: run IDs specified in BOX 3

\Link{B}{after}{A}{10}
\Link{C}{after}{B}{}
%%%%%%%%%%%%%%%%%%%%%%%%%%%%%%%%%%%%%%%%%%%%%%%%%%%%%%%%%%%%%%%%%%%%%%%%
%
%---- BOX 14 -----------------------------------------------------------
%
% INSTRUMENT CONFIGURATIONS:
%
% Uncomment only the lines related to instrument configuration(s)
% needed for the acquisition of your planned observations.
%
% 1st argument: run ID
% Valid values: run IDs specified in BOX 3
%
% 2nd argument: instrument
% This parameter is NOT checked at the pdfLaTeX compilation.
%
% 3rd argument: mode
% This parameter is NOT checked at the pdfLaTeX compilation.
% Please note that RRM mode is only available for some specific
% instrument configurations.
%
% 4th argument: additional information
% This parameter is NOT checked at the pdfLaTeX compilation.
%
% All parameters are mandatory and cannot be empty. Do NOT specify
% Instrument Configurations for alternative runs.

% Examples (to be commented or deleted)

\INSconfig{A}{FORS2}{Detector}{MIT}
\INSconfig{A}{FORS2}{IMG}{ESO filters: provide list HERE}
\INSconfig{B}{XSHOOTER}{SLT}{readout UVB,readout VIS,readout NIR}
\INSconfig{C}{EFOSC2}{Imaging-filters}{EFOSC2 filters: provide list here}
\INSconfig{D}{XSHOOTER}{SLT}{readout UVB,readout VIS,readout NIR}
\INSconfig{E}{XSHOOTER}{SLT}{readout UVB,readout VIS,readout NIR}
%
% Real list of instrument configurations
%
%%%%%%%%%%%%%%%%%%%%%%%%%%%%%%%%%%%%%%%%%%%%%%%%%%%%%%%%%%%%%%%%%%%%%%%%%
% Paranal
%
%-----------------------------------------------------------------------
%---- FORS2 at the VLT-UT1 (ANTU) --------------------------------------
%-----------------------------------------------------------------------
%
%If you require the E2V (Blue) detector uncomment the following line
%\INSconfig{}{FORS2}{Detector}{E2V}
%
%If you require the MIT (RED) detector uncomment the following line
%\INSconfig{}{FORS2}{Detector}{MIT}
%
% If you require the High-Resolution  collimator uncomment the following line
%\INSconfig{}{FORS2}{collimator}{HR}
%
% Uncomment the line(s) corresponding to the imaging mode(s) you require and
% provide the list of filters needed  for your observations:
%
%\INSconfig{}{FORS2}{PRE-IMG}{ESO filters: provide list HERE}
%\INSconfig{}{FORS2}{IMG}{ESO filters: provide list HERE}
%\INSconfig{}{FORS2}{IMG}{User's own filters (to be described in text)}
%\INSconfig{}{FORS2}{IPOL}{ESO filters: provide list HERE}
%\INSconfig{}{FORS2}{IPOL}{User's own filters (to be described in text)}
%
% Uncomment the line(s) corresponding to the spectroscopic mode(s) you require and
% provide the list of grism+filter combination needed  for your observations:
%
%\INSconfig{}{FORS2}{LSS}{Provide list of grism+filter combinations HERE}
%\INSconfig{}{FORS2}{MOS}{Provide list of grism+filter combinations HERE}
%\INSconfig{}{FORS2}{PMOS}{Provide list of grism+filter combinations HERE}
%\INSconfig{}{FORS2}{MXU}{Provide list of grism+filter combinations HERE}
%
% Uncomment the following line for Rapid Response Mode observations
%
%\INSconfig{}{FORS2}{RRM}{yes}
%
% Uncomment the following line for use of the Virtual Image Slicer
%\INSconfig{}{FORS2}{Virtual Image Slicer}{VM only}
%
%-----------------------------------------------------------------------
%---- KMOS at the VLT-UT1 (ANTU) ---------------------------------------
%-----------------------------------------------------------------------
%
%\INSconfig{}{KMOS}{IFU}{provide list of settings (IZ, YJ, H, K, HK) here} 
%
%-----------------------------------------------------------------------
%---- FLAMES at the VLT-UT2 (KUEYEN) -----------------------------------
%-----------------------------------------------------------------------
%
%\INSconfig{}{FLAMES}{UVES}{Specify the UVES setup below}
%\INSconfig{}{FLAMES}{GIRAFFE-MEDUSA}{Specify the GIRAFFE setup below}
%\INSconfig{}{FLAMES}{GIRAFFE-IFU}{Specify the GIRAFFE setup below}
%\INSconfig{}{FLAMES}{GIRAFFE-ARGUS}{Specify the GIRAFFE setup below}
%\INSconfig{}{FLAMES}{Combined: UVES + GIRAFFE-MEDUSA}{Specify the UVES and GIRAFFE setups below}
%\INSconfig{}{FLAMES}{Combined: UVES + GIRAFFE-IFU}{Specify the UVES and GIRAFFE setups below}
%\INSconfig{}{FLAMES}{Combined: UVES + GIRAFFE-ARGUS}{Specify the UVES and GIRAFFE setups below}
%
% If you have selected UVES, either standalone or in combined mode,
% please specify the UVES standard setup(s) to be used
%\INSconfig{}{FLAMES}{UVES}{standard setup Red 520}
%\INSconfig{}{FLAMES}{UVES}{standard setup Red 580}
%\INSconfig{}{FLAMES}{UVES}{standard setup Red 580 + simultaneous calibration}
%\INSconfig{}{FLAMES}{UVES}{standard setup Red 860}
%
%\INSconfig{}{FLAMES}{GIRAFFE}{fast readout mode 625kHz VM only}
%\INSconfig{}{FLAMES}{GIRAFFE}{slow readout mode 50kHz VM only}
%
% If you have selected GIRAFFE, either standalone or in combined mode
% please specify the GIRAFFE standard setups(s) to be used
%\INSconfig{}{FLAMES}{GIRAFFE}{standard setup HR01 379.0}
%\INSconfig{}{FLAMES}{GIRAFFE}{standard setup HR02 395.8}
%\INSconfig{}{FLAMES}{GIRAFFE}{standard setup HR03 412.4}
%\INSconfig{}{FLAMES}{GIRAFFE}{standard setup HR04 429.7}
%\INSconfig{}{FLAMES}{GIRAFFE}{standard setup HR05 447.1 A}
%\INSconfig{}{FLAMES}{GIRAFFE}{standard setup HR05 447.1 B}
%\INSconfig{}{FLAMES}{GIRAFFE}{standard setup HR06 465.6}
%\INSconfig{}{FLAMES}{GIRAFFE}{standard setup HR07 484.5 A}
%\INSconfig{}{FLAMES}{GIRAFFE}{standard setup HR07 484.5 B}
%\INSconfig{}{FLAMES}{GIRAFFE}{standard setup HR08 504.8}
%\INSconfig{}{FLAMES}{GIRAFFE}{standard setup HR09 525.8 A}
%\INSconfig{}{FLAMES}{GIRAFFE}{standard setup HR09 525.8 B}
%\INSconfig{}{FLAMES}{GIRAFFE}{standard setup HR10 548.8}
%\INSconfig{}{FLAMES}{GIRAFFE}{standard setup HR11 572.8}
%\INSconfig{}{FLAMES}{GIRAFFE}{standard setup HR12 599.3}
%\INSconfig{}{FLAMES}{GIRAFFE}{standard setup HR13 627.3}
%\INSconfig{}{FLAMES}{GIRAFFE}{standard setup HR14 651.5 A}
%\INSconfig{}{FLAMES}{GIRAFFE}{standard setup HR14 651.5 B}
%\INSconfig{}{FLAMES}{GIRAFFE}{standard setup HR15 665.0}
%\INSconfig{}{FLAMES}{GIRAFFE}{standard setup HR15 679.7}
%\INSconfig{}{FLAMES}{GIRAFFE}{standard setup HR16 710.5}
%\INSconfig{}{FLAMES}{GIRAFFE}{standard setup HR17 737.0 A}
%\INSconfig{}{FLAMES}{GIRAFFE}{standard setup HR17 737.0 B}
%\INSconfig{}{FLAMES}{GIRAFFE}{standard setup HR18 769.1}
%\INSconfig{}{FLAMES}{GIRAFFE}{standard setup HR19 805.3 A}
%\INSconfig{}{FLAMES}{GIRAFFE}{standard setup HR19 805.3 B}
%\INSconfig{}{FLAMES}{GIRAFFE}{standard setup HR20 836.6 A}
%\INSconfig{}{FLAMES}{GIRAFFE}{standard setup HR20 836.6 B}
%\INSconfig{}{FLAMES}{GIRAFFE}{standard setup HR21 875.7}
%\INSconfig{}{FLAMES}{GIRAFFE}{standard setup HR22 920.5 A}
%\INSconfig{}{FLAMES}{GIRAFFE}{standard setup HR22 920.5 B}
%\INSconfig{}{FLAMES}{GIRAFFE}{standard setup LR01 385.7}
%\INSconfig{}{FLAMES}{GIRAFFE}{standard setup LR02 427.2}
%\INSconfig{}{FLAMES}{GIRAFFE}{standard setup LR03 479.7}
%\INSconfig{}{FLAMES}{GIRAFFE}{standard setup LR04 543.1}
%\INSconfig{}{FLAMES}{GIRAFFE}{standard setup LR05 614.2}
%\INSconfig{}{FLAMES}{GIRAFFE}{standard setup LR06 682.2}
%\INSconfig{}{FLAMES}{GIRAFFE}{standard setup LR07 773.4}
%\INSconfig{}{FLAMES}{GIRAFFE}{standard setup LR08 881.7}
%
%\INSconfig{}{FLAMES}{GIRAFFE}{fast readout mode 625kHz VM only}
%
%-----------------------------------------------------------------------
%---- X-SHOOTER at the VLT-UT2 (KUEYEN)
%-----------------------------------------------------------------------
%
%\INSconfig{}{XSHOOTER}{300-2500nm}{SLT}
%\INSconfig{}{XSHOOTER}{300-2500nm}{IFU}
%
% Slits (SLT only):
%
%UVB arm, available slits in arcsec: 0.5, 0.8, 1.0, 1.3, 1.6, 5.0
%VIS arm, available slits in arcsec: 0.4, 0.7, 0.9, 1.2, 1.5, 5.0 
%NIR arm, available slits in arcsec: 0.4, 0.6, 0.6JH, 0.9, 0.9JH, 1.2, 5.0
%  The 0.6JH and 0.9JH include a stray light K-band blocking filter
%  that allow sky limited studies in J and H bands.
%
%The slits for IFU  are fixed and do not need to be mentioned here.
%
% Replace SLIT-UVB, SLIT-VIS, SLIT-NIR with the choice of the slits:
%\INSconfig{}{XSHOOTER}{SLT}{SLIT-UVB,SLIT-VIS,SLIT-NIR}
%
% Detector readout mode:
%
% UVB and VIS arms: available readout modes and binning:
% 100k-1x1, 100k-1x2, 100k-2x2, 400k-1x1, 400k-1x2, 400k-2x2
% The NIR readout mode is fixed  to NDR.
%
%\INSconfig{}{XSHOOTER}{IFU}{readout UVB,readout VIS,readout NIR}
%\INSconfig{}{XSHOOTER}{SLT}{readout UVB,readout VIS,readout NIR}
%
% Imaging mode 
% replace 'list of filters' by the actual filters you wish to use among:
% U, B, V, R, I, Uprime, Gprime, Rprime, Iprime, Zprime
% Please note that the imaging mode can only be used in combination with slit or IFU observations
%\INSconfig{}{XSHOOTER}{IMG}{list of filters}
%
%\INSconfig{}{XSHOOTER}{RRM}{yes}
%
% Uncomment the following line for use of the Virtual Image Slicer
%\INSconfig{}{XSHOOTER}{Virtual Image Slicer}{VM only}
%
%-----------------------------------------------------------------------
%---- UVES at the VLT-UT2 (KUEYEN) -------------------------------------
%-----------------------------------------------------------------------
%
%\INSconfig{}{UVES}{BLUE}{Standard setting: 346}
%\INSconfig{}{UVES}{BLUE}{Standard setting: 437}
%\INSconfig{}{UVES}{BLUE}{Non-std setting: provide central wavelength  HERE}
%
%\INSconfig{}{UVES}{RED}{Standard setting: 520}
%\INSconfig{}{UVES}{RED}{Standard setting: 580}
%\INSconfig{}{UVES}{RED}{Standard setting: 600}
%\INSconfig{}{UVES}{RED}{Iodine cell standard setting: 600}
%\INSconfig{}{UVES}{RED}{Standard setting: 860}
%\INSconfig{}{UVES}{RED}{Non-std setting: provide central wavelength HERE}
%
%\INSconfig{}{UVES}{DIC-1}{Standard setting: 346+580}
%\INSconfig{}{UVES}{DIC-1}{Standard setting: 390+564}
%\INSconfig{}{UVES}{DIC-1}{Standard setting: 346+564}
%\INSconfig{}{UVES}{DIC-1}{Standard setting: 390+580}
%\INSconfig{}{UVES}{DIC-1}{Non-std setting: provide central wavelength HERE}
%
%\INSconfig{}{UVES}{DIC-2}{Standard setting: 437+860}
%\INSconfig{}{UVES}{DIC-2}{Standard setting: 346+860}
%\INSconfig{}{UVES}{DIC-2}{Standard setting: 390+860}
%
%\INSconfig{}{UVES}{DIC-2}{Standard setting: 437+760}
%\INSconfig{}{UVES}{DIC-2}{Standard setting: 346+760}
%\INSconfig{}{UVES}{DIC-2}{Standard setting: 390+760}
%\INSconfig{}{UVES}{DIC-2}{Non-std setting: provide central wavelength HERE}
%
%\INSconfig{}{UVES}{Field Derotation}{yes}
%\INSconfig{}{UVES}{Image slicer-1}{yes}
%\INSconfig{}{UVES}{Image slicer-2}{yes}
%\INSconfig{}{UVES}{Image slicer-3}{yes}
%\INSconfig{}{UVES}{Iodine cell}{yes}
%\INSconfig{}{UVES}{Longslit Filters}{Provide list of filters HERE}
%
%\INSconfig{}{UVES}{RRM}{yes}
%
% Uncomment the following line for use of the Virtual Image Slicer
%\INSconfig{}{UVES}{Virtual Image Slicer}{VM only}
%
%-----------------------------------------------------------------------
%---- SPHERE at the VLT-UT3 (MELIPAL) -----------------------------------
%-----------------------------------------------------------------------
%
% Pupil or field tracking?
% Mode choices: IRDIS-CI, IRDIS-DBI, 
%               IRDIFS, IRDIFS-EXT, 
%               ZIMPOL-I
%               (Not relevant for IRDIS-DPI, IRDIS-LSS, ZIMPOL-P1 or ZIMPOL-P2)
%--------------------
% IRDIFS: 
% Coronagraph combination choices:
%   IRDIFS:     None, N-ALC-YJH-S, N-ALC-YJH-L, N-CLC-SW-L, N-SAM-7H
%   IRDIFS-EXT: None, N-ALC-YJH-S, N-ALC-YJH-L, N-ALC-Ks, N-SAM-7H
% Filter choices for IRDIS in IRDIFS mode
%   IRDIFS:     DB-H23, DB-ND23, DB-H34, BB-H
%   IRDIFS-EXT: DB-K12, BB-Ks
%---------------------
% IRDIS imaging (alone):
% Coronagraph combination choices for IRDIS imaging modes (see UM for details)
%   IRDIS-CI, IRDIS-DPI:  
%              None, N-ALC-Y, N-ALC-YJ-S, N-ALC-YJ-L, N-ALC-YJH-S, 
%                    N-ALC-YJH-L, N-ALC-Ks, N-SAM-7H
%   IRDIS-DBI: None, N-ALC-Y, N-ALC-YJ-S, N-ALC-YJ-L, N-ALC-YJH-S, 
%                    N-ALC-YJH-L, N-ALC-Ks, N-SAM-7H
% Filter choices:
%   IRDIS-CI, IRDIS-DPI: 
%              BB-Y, BB-J, BB-H, BB-Ks, NB-Hel, NB-CntJ, NB-CntH,
%              NB-CntK1, NB-BrG, NB-CntK2, NB-PaB, NB-FeII, NB-H2, NB-CO
%   IRDIS-DBI: DB-Y23, DB-J23, DB-H23, DB-NDH23,  DB-H34, DB-K12 
%---------------------
% IRDIS spectroscopy:
% Coronagraphic slit/grism combinations for IRDIS-LSS:
%   IRDIS-LSS: N-S-LR-WL, N-S-MR-WL, 
%              N_S_APO_LR_WL, N_S_APO_MR_WL, N_S_APO_MR_NL
%---------------------
% ZIMPOL imaging: 
% Coronagraph choices:
%   ZIMPOL-I: None, V-CLC-M-WF, V-CLC-M-NF, V-CLC-L-WF, V-CLC-XL-WF, V-SAM-7H
% Filter choices:
%   ZIMPOL-I: RI, R-PRIM, I-PRIM, V, V-S, V-L, N-R, 730-NB, N-I, I-L,
%             KI,  TiO-717, CH4-727, Cnt748, Cnt820, HeI, OI-630,
%             CntHa, B-Ha, N-Ha, Ha-NB
%--------------------
% ZIMPOL polarimetry:
% Coronagraph choices:
%    ZIMPOL-P1: None, V-CLC-S-WF, V-CLC-M-WF, V-CLC-L-WF, V-CLC-XL-WF, V-CLC-MT-WF
%    ZIMPOL-P2: None, V-CLC-S-WF, V-CLC-M-WF, V-CLC-L-WF, V-CLC-XL-WF, V-CLC-MT-WF
% Filter choices:
%    ZIMPOL-P1: RI, R-PRIM, I-PRIM, V, N-R, N-I, KI, TiO-717, 
%               CH4-727, Cnt748, Cnt820, CntHa, N-Ha, B-Ha     
%    ZIMPOL-P2: RI, R-PRIM, I-PRIM, V, N-R, N-I, KI, TiO-717, 
%               CH4-727, Cnt748, Cnt820, CntHa, N-Ha, B-Ha 
% Readout mode choice for ZIMPOL
%    ZIMPOL-P1: FastPol, SlowPol
%    ZIMPOL-P2: FastPol, SlowPol
%-------------------
%
% One entry per mode. Repeat the entry for each mode.
%
%\INSconfig{}{SPHERE}{Pupil}{mode}
%\INSconfig{}{SPHERE}{Field}{mode}
%
% One entry per combination. Repeat the entry for each combination.
%
%\INSconfig{}{SPHERE}{IRDIFS}{Coronagraph/filter or SAM mask combination for IRDIFS}
%\INSconfig{}{SPHERE}{IRDIFS-EXT}{Coronagraph/filter or SAM mask combination for IRDIFS-EXT}
%
%\INSconfig{}{SPHERE}{IRDIS-CI}{Coronagraph/filter or SAM mask combination for IRDIS-CI}
%\INSconfig{}{SPHERE}{IRDIS-DBI}{Coronagraph/filter or SAM mask  combination for IRDIS-DBI}
%\INSconfig{}{SPHERE}{IRDIS-DPI}{Coronagraph/filter or SAM mask combination for IRDIS-DPI}
%\INSconfig{}{SPHERE}{IRDIS-LSS}{Coronagraphic slit/grism combination for IRDIS-LSS}
%
%\INSconfig{}{SPHERE}{ZIMPOL-I}{Coronagraph/filter or SAM mask combination for ZIMPOL-I}
%
%\INSconfig{}{SPHERE}{ZIMPOL-P1}{Coronagraph/filter/readout mode for ZIMPOL-P1}
%\INSconfig{}{SPHERE}{ZIMPOL-P2}{Coronagraph/filter/readout mode for ZIMPOL-P2}
%
%------------------  
% Uncomment the following line if the run requires the Rapid Response Mode
% 'Mode' should be one of IRDIFS, IRDIFS-EXT, IRDIS-CI, IRDIS-DBI,
% IRDIS-DPI, IRDIS-LSS, ZIMPOL-I, ZIMPOL-P1, ZIMPOLP2,
%
%\INSconfig{}{SPHERE}{RRM}{Mode}
% 
%-----------------------------------------------------------------------
%---- VISIR at the VLT-UT3/UT4 (MELIPAL) -----------------------------------
%-----------------------------------------------------------------------
%
% List of offered filters for IMG:
%    M-BAND, J7.9, PAH1, J8.9, B8.7, ArIII, J9.8, SIV-1, B9.7, SIV, B10.7,
%    SIV-2, PAH2, B11.7, PAH2-2, J12.2, NeII-1, B12.4, NeII, NeII-2, Q1, Q2, Q3
%
%\INSconfig{}{VISIR}{IMG 45 mas/px}{Provide list of filters HERE}
%\INSconfig{}{VISIR}{IMG 76 mas/px}{Provide list of filters HERE}
%
% List of offered filters for CORONA AGPM:
%    10-5-4QP,11-3-4QP,12-3-AGP
%\INSconfig{}{VISIR}{CORONA 45 mas/px}{List of filters}
%
% List of filters offered for SAM:
%    10-5-SAM, 11-3-SAM
%
%\INSconfig{}{VISIR}{SAM 45 mas/px}{List of filters}
%
% Spectroscopy:
%
%\INSconfig{}{VISIR}{SPEC N-band LR}{-}
%\INSconfig{}{VISIR}{SPEC N-band HR Longslit}{Provide central wavelengt(s) (8.02,12.81) HERE}
%\INSconfig{}{VISIR}{SPEC Q-band HR Longslit}{Provide central wavelength(s) (17.03) HERE}
%\INSconfig{}{VISIR}{SPEC N-band HRCrossdispersed}{Provide central wavelength(s) (7.7-13.3)}
%\INSconfig{}{VISIR}{SPEC Q-band HRCrossdispersed}{Provide central wavelength(s) (16.0-24.0) HERE}
%
%-----------------------------------------------------------------------
%---- HAWKI at the VLT-UT4 (YEPUN) -----------------------------------
%-----------------------------------------------------------------------
%
%\INSconfig{}{HAWKI}{PRE-IMG}{provide list of filters (Y,J,H,Ks,CH4,BrG,H2,NB1190,NB1060,NB2090) HERE}
%
%\INSconfig{}{HAWKI}{IMG}{provide list of filters (Y,J,H,Ks,CH4,BrG,H2,NB1190,NB1060,NB2090) HERE}
%\INSconfig{}{HAWKI}{FASTJITT}{Provide list of filters  (Y,J,H,Ks,CH4,BrG,H2,NB1190,NB1060,NB2090) HERE}
%
%\INSconfig{}{HAWKI}{AO-IMG}{provide list of filters (Y,J,H,Ks,CH4,BrG,H2,NB1190,NB1060,NB2090) HERE}
%\INSconfig{}{HAWKI}{AO-FASTJITT}{Provide list of filters  (Y,J,H,Ks,CH4,BrG,H2,NB1190,NB1060,NB2090) HERE}
%
% Uncomment for RRM (NoAO only)
%\INSconfig{}{HAWKI}{RRM}{yes}
%
%-----------------------------------------------------------------------
%---- MUSE at the VLT-UT4 (YEPUN) -----------------------------------
%-----------------------------------------------------------------------
%
% If you plan to use MUSE in NOAO mode, please uncomment one of these lines.
%\INSconfig{}{MUSE}{WFM-NOAO-N}{-}
%\INSconfig{}{MUSE}{WFM-NOAO-E}{-}
%
% If you plan to use the LGS, please specify the Tip-Tilt star name and magnitude (Rmag preferred,
% otherwise Vmag) in the target list.
%\INSconfig{}{MUSE}{WFM-AO-N LGS}{-}
%\INSconfig{}{MUSE}{WFM-AO-E LGS}{-}
%
% If you plan to use the LGS in WFM without a Tip-Tilt star (seeing enhancer mode)
%\INSconfig{}{MUSE}{WFM-AO-N LGS-noTTS}{-}
%\INSconfig{}{MUSE}{WFM-AO-E LGS-noTTS}{-}
%
% If you plan to use the LGS in NFM, please specify the Tip-Tilt star
% name and infrared H-band magnitude in the target list
%\INSconfig{}{MUSE}{NFM-AO-N LGS}{-}
%
% Uncomment the following line for Rapid Response Mode observations (NOAO mode only)
%\INSconfig{}{MUSE}{RRM}{yes}
%
%-----------------------------------------------------------------------
%---- ESPRESSO at the VLT-ICCF -----------------------------------------
%-----------------------------------------------------------------------
%
% ESPRESSO in 1 UT mode:
%
%\INSconfig{}{ESPRESSO-1UT}{HR}{1x1, 2x1}
%\INSconfig{}{ESPRESSO-1UT}{UHR}{1x1}
%
%-----------------------------------------------------------------------
%---- GRAVITY ----------------------------------------------------------
%-----------------------------------------------------------------------
%
%\INSconfig{}{GRAVITY}{Single-Field}{provide list of grating(s) (LR,MR,HR) HERE}
%\INSconfig{}{GRAVITY}{Dual-Field}{provide list of grating(s) (LR,MR,HR) HERE}
%\INSconfig{}{GRAVITY}{Astrometry}{provide list of grating(s)(LR,MR,HR) HERE}
%
%%For UT runs, uncomment the following line and specify the Wave Front Sensor to be used: MACAO or CIAO, on- or off-axis:
%\INSconfig{}{GRAVITY}{WFS}{MACAO-ON, MACAO-OFF, CIAO-ON or CIAO-OFF}
%%
%% Please uncomment one or several observation types for each run:
%%\INSconfig{}{GRAVITY}{provide observation type HERE (snapshot, time series, imaging, astrometry)}{-}
%
%----------------------------------------------------------------------
%---- MATISSE at the VLTI ---------------------------------------------
%----------------------------------------------------------------------
%% For L band the dispersive element must be specified.  For N
% band in P103 only the low resolution is available and always
% observed in parallel to L band. 
%
%
%\INSconfig{}{MATISSE}{L}{Give list of dispersive elements to be used (LR,MR,HR)}
%\INSconfig{}{MATISSE}{N}{Give list of dispersive elements to be used (LR,HR)}
%
%% Please uncomment one or several observation types for each run:
%\INSconfig{}{MATISSE}{provide observation type HERE (snapshot, time series, imaging, astrometry)}{-}
%
%-----------------------------------------------------------------------
%---- PIONIER ----------------------------------------------------------
%-----------------------------------------------------------------------
%
%\INSconfig{}{PIONIER}{GRISM}{1.65}
%\INSconfig{}{PIONIER}{FREE}{1.65}
%
%% Please uncomment one or several observation types for each run:
%\INSconfig{}{PIONIER}{provide observation type HERE (snapshot, time series, imaging, astrometry)}{-}
%
%-----------------------------------------------------------------------
%---- VIRCAM at VISTA --------------------------------------------------
%-----------------------------------------------------------------------
%
%\INSconfig{}{VIRCAM}{IMG}{provide list of filters here}
%
%-----------------------------------------------------------------------
%---- OMEGACAM at VST --------------------------------------------------
%-----------------------------------------------------------------------
%
%\INSconfig{}{OMEGACAM}{IMG}{provide list of filters here}
%
%%%%%%%%%%%%%%%%%%%%%%%%%%%%%%%%%%%%%%%%%%%%%%%%%%%%%%%%%%%%%%%%%%%%%%%%
% La Silla
%
%-----------------------------------------------------------------------
%---- HARPS at the 3.6 -------------------------------------------------
%-----------------------------------------------------------------------
%
%\INSconfig{}{HARPS}{spectro-Thosimult}{HARPS}
%\INSconfig{}{HARPS}{WAVE}{HARPS}
%\INSconfig{}{HARPS}{spectro-ObjA(B)}{HARPS}
%\INSconfig{}{HARPS}{spectro-ObjA(B)}{EGGS}
%\INSconfig{}{HARPS}{spectro-polarimetry}{linear}
%\INSconfig{}{HARPS}{spectro-polarimetry}{circular}
%
%%%%%%%%%%%%%%%%%%%%%%%%%%%%%%%%%%%%%%%%%%%%%%%%%%%%%%%%%%%%%%%%%%%%%%%%
% Chajnantor
%
%-----------------------------------------------------------------------
%---- ARTEMIS at APEX ----------------------------------------------
%-----------------------------------------------------------------------
%
%\INSconfig{}{ARTEMIS}{IMG}{350 and 450 um}
%
%-----------------------------------------------------------------------
%---- SEPIA at APEX ----------------------------------------------
%-----------------------------------------------------------------------
%
%\INSconfig{}{SEPIA}{SEPIA180}{Please enter Central Frequency 159 to 211 GHz}
%\INSconfig{}{SEPIA}{SEPIA345}{Please enter Central Frequency 272 to 376 GHz}
%\INSconfig{}{SEPIA}{SEPIA660}{Please enter Central Frequency 602 to 720 GHz}
%

%%%%%%%%%%%%%%%%%%%%%%%%%%%%%%%%%%%%%%%%%%%%%%%%%%%%%%%%%%%%%%%%%%%%%%%%
%%%%% Interferometry PAGE %%%%%%%%%%%%%%%%%%%%%%%%%%%%%%%%%%%%%%%%%%%%%%
%%%%%%%%%%%%%%%%%%%%%%%%%%%%%%%%%%%%%%%%%%%%%%%%%%%%%%%%%%%%%%%%%%%%%%%%
%
% The \VLTITarget macro is only needed when requesting
% interferometry, in which case it is MANDATORY to uncomment it and
% fill in the information. It takes the following parameters:
%
% 1st argument: run ID
% Valid values: run IDs specified in BOX 3
%
% 2nd argument: target name
% This parameter is NOT checked at the pdfLaTeX compilation.
%
% 3rd argument: visual magnitude
% Values with up to two decimal places are allowed here.
% This parameter is NOT checked at the pdfLaTeX compilation.
%
% 4th argument: magnitude at wavelength of observation
% Values with up to two decimal places are allowed here.
% This parameter is NOT checked at the pdfLaTeX compilation.
%
% 5th argument: wavelength of observation (in microns)
% Values with up to two decimal places are allowed here.
% This parameter is NOT checked at the pdfLaTeX compilation.
%
% 6th argument: size at wavelength of observation (in mas)
% This parameter is NOT checked at the pdfLaTeX compilation.
%
% 7th argument: baseline
% UT observations are scheduled in terms of 4-telescope baselines for PIONIER, MATISSE and GRAVITY.
%
% AT observations are scheduled in terms of 4-telescope
% configurations (quadruplets) for any instrument. 
% For AT observations with any instrument, please specify 
% one of the AT quadruplets available for a given instrument/mode.
%
% 8th parameter: Range of visibilities for the specified configuration.
% Please specify the maximum and minimum visibility values
% corresponding to the chosen configuration at hour angle 0
% separated by "/".
% This parameter is NOT checked at the pdfLaTeX compilation. 
%
% 9th parameter: correlated magnitude
% (for the visibility values specified in the 8th parameter)
% This parameter is NOT checked at the pdfLaTeX compilation.
%
% 10th parameter: time on target in hours
% Values with up to two decimal places are allowed here.
% This parameter is NOT checked at the pdfLaTeX compilation.
% 
% PIONIER
% Large:       corresponding to the A0-G1-J2-J3 configuration
% Medium:      corresponding to the D0-G2-J3-K0 configuration
% Small:       corresponding to the A0-B2-C1-D0 configuration
% UTs:         corresponding to the UT1-UT2-UT3-UT4 configuration
% 
% MATISSE
% Large:       corresponding to the A0-G1-J2-J3 configuration
% Medium:      corresponding to the D0-G2-J3-K0 configuration
% Small:       corresponding to the A0-B2-C1-D0 configuration
% UTs:         corresponding to the UT1-UT2-UT3-UT4 configuration
%
% GRAVITY
% Small:       corresponding to the A0-B2-C1-D0 configuration, offered in both single-field and dual-field modes
% Medium:      corresponding to the D0-G2-J3-K0 configuration, only offered in single-field mode 
% Large:       corresponding to the A0-G1-J2-J3 configuration, only offered in single-field mode 
% Astrometric: corresponding to the A0-G1-J2-K0 configuration, only offered in dual-field mode 
% UTs:         corresponding to the UT1-UT2-UT3-UT4 configuration
%
%\VLTITarget{E}{Alpha Ori}{-1.4}{-1.4}{10.6}{6}{UTs}{0.60/0.10}{-0.2/4.0}{2} 
%\VLTITarget{F}{Alpha Ori}{-1.4}{-1.4}{10.6}{6}{Large}{0.80/0.40}{-0.9/-0.2}{1} 

% You can specify here a note applying to all or some of your VLTI
% targets. You should take advantage of this note to indicate
% suitable alternative baselines for your observations.
% This macro is NOT checked at the pdfLaTeX compilation.

%\VLTITargetNotes{Note about the VLTI targets, e.g., Run F can also be carried out using the astrometric configuration.}
%
%%%%%%%%%%%%%%%%%%%%%%%%%%%%%%%%%%%%%%%%%%%%%%%%%%%%%%%%%%%%%%%%%%%%%%%%
%%%%% ToO PAGE %%%%%%%%%%%%%%%%%%%%%%%%%%%%%%%%%%%%%%%%%%%%%%%%%%%%%%%%%
%%%%%%%%%%%%%%%%%%%%%%%%%%%%%%%%%%%%%%%%%%%%%%%%%%%%%%%%%%%%%%%%%%%%%%%%
%
% The \ToOrun macro is needed only when requesting Target of
% Opportunity (ToO) observations, in which case it is MANDATORY to
% uncomment it and fill in the information. It takes the following
% parameters: 
%
% 1st argument: run ID
% Valid values: run IDs specified in BOX 3
%
% 2nd argument: nature of observation
% Valid values: ToO-hard, ToO-soft
%
% 3rd argument: number of targets per run
% This parameter is NOT checked at the pdfLaTeX compilation.
%
% 4th argument: number of triggers per targets
% (for ToO observations only)
% This parameter is NOT checked at the pdfLaTeX compilation.

%\TOORun{A}{ToO-hard}{2}{3}
%\TOORun{B}{ToO-soft}{3}{1}

% You have the opportunity to add notes to the ToO runs by using
% the \TOONotes macro.
% This macro is NOT checked at the pdfLaTeX compilation.

%\TOONotes{Use this macro to add a note to the ToO page.}

%%%%%%%%%%%%%%%%%%%%%%%%%%%%%%%%%%%%%%%%%%%%%%%%%%%%%%%%%%%%%%%%%%%%%%%%
%%%%% VISITOR SPECIAL INSTRUMENT PAGE %%%%%%%%%%%%%%%%%%%%%%%%%%%%%%%%%%
%%%%%%%%%%%%%%%%%%%%%%%%%%%%%%%%%%%%%%%%%%%%%%%%%%%%%%%%%%%%%%%%%%%%%%%%
%
% The following commands are only needed when bringing a Visitor
% Special Instrument, in which case it is MANDATORY to uncomment them
% and provide all the required information.
%
%\Desc{}   %Description of the instrument and its operation
%\Comm{}   %On which telescope(s) has instrument been commissioned/used
%\WV{}     %Total weight and value of equipment to be shipped
%\Wfocus{} %Weight at the focus (including ancillary equipment)
%\Interf{} %Compatibility of attachment interface with required focus
%\Focal{}  %Back focal distance value
%\Acqu{}   %Acquisition, focusing, and guiding procedure
%\Softw{}  %Compatibility with ESO software standards (data handling)
%\Suppl{}  %Estimate of services expected from ESO (in person days)

%%%%%%%%%%%%%%%%%%%%%%%%%%%%%%%%%%%%%%%%%%%%%%%%%%%%%%%%%%%%%%%%%%%%%%%%
%%%%% THE END %%%%%%%%%%%%%%%%%%%%%%%%%%%%%%%%%%%%%%%%%%%%%%%%%%%%%%%%%%
%%%%%%%%%%%%%%%%%%%%%%%%%%%%%%%%%%%%%%%%%%%%%%%%%%%%%%%%%%%%%%%%%%%%%%%%
\MakeProposal
\end{document}


